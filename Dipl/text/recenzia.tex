\documentclass[14pt,a4paper,titlepage]{extarticle}
\usepackage[T2A,T1]{fontenc}
\usepackage[utf8]{inputenc}
\usepackage[ukrainian]{babel}
\usepackage{pgfplots}
\usepackage{hyperref}
\usepackage[margin=0.6in,top=4cm,bottom=4cm]{geometry}
\usepackage{setspace}

\pagenumbering{gobble}

\begin{document}

\begin{center}
\setstretch{1.2}
\textbf{РЕЦЕНЗІЯ \\[0.7cm]
на випускну кваліфікаційну роботу бакалавра \\
«Числові методи аналізу ДНК» \\
студента 4-го курсу кафедри обчислювальної математики \\
факультету кібернетики Київського національного \\
університету імені Тараса Шевченка \\
Товта Аттіли Аттіловича
}
\end{center}
\setstretch{1.6}
На сьогодні проблема порівняльної характеристики ДНК є актуальною. За останні роки кількість зразків ДНК росте експотенційно, через це важливо розробити методи які могли б швидко і якісно розв’язати цю проблему.
\par
У роботі розглянуто декілька  різних способів числового подання послідовності ДНК з подальшим застосуванням $p$-статистик. При аналізі алгоритмів використовувались реальні послідовності ДНК.
\par
Істотних недоліків у роботі не виявлено. Дипломна робота має практичне значення. В цілому роботу виконано на високому науковому рівні.
\par
Розкрита тема роботи, продемонстровано високий рівень кваліфікації. Робота заслуговує оцінки «відмінно», а автор Товт А. А. - присвоєння кваліфікації бакалавра прикладної математики.

\vfill

\setstretch{1.0}
\begin{minipage}[t]{11cm}
\flushleft
Член-кореспондент НАН України, \\
завідувач кафедри обчислювальної математики, \\
доктор фізико-математичних наук, професор
\end{minipage}
\begin{minipage}[t]{6cm}
\hfill
\flushright
Ляшко С. І.
\end{minipage}

\end{document}

