\documentclass[14pt,a4paper,titlepage]{extarticle}
\usepackage[T2A,T1]{fontenc}
\usepackage[utf8]{inputenc}
\usepackage[ukrainian]{babel}
\usepackage{pgfplots}
\usepackage{hyperref}
\usepackage[margin=0.6in,top=4cm,bottom=4cm]{geometry}
\usepackage{setspace}

\pagenumbering{gobble}

\begin{document}

\begin{center}
\setstretch{1.2}
\textbf{ВІДГУК \\[0.7cm]
на випускну кваліфікаційну роботу бакалавра \\
«Числові методи аналізу ДНК» \\
студента 4-го курсу кафедри обчислювальної математики \\
факультету кібернетики Київського національного \\
університету імені Тараса Шевченка \\
Товта Аттіли Аттіловича
}
\end{center}
\setstretch{1.6}
З збільшенням кількості послідовностей ДНК і розвитком біотехнологій,
з’явилась необхідність порівнювати послідовності ДНК. Було запропоновано
багато обчислювальних і статистичних методів порівняння генетичних послідовностей, але незважаючи на це, тема залишається актуальної і на цей час.
\par
У випускній роботі розглянуто сім методів числового задання ДНК і застосування $p$-статистик для їх порівняння. Студентом була написана програма, яка обчислює за заданим відрізком ДНК його чисельне представлення і $p$-статистики. Також використовувалися реальні послідовності ДНК. Крім того для кращого розуміння і тестування була написана прорама, яка візуалізовувала деякі методи чисельного представлення послідовності ДНК.
\par
Робота Товт А. А. виконана на високому рівні, тема розкрита повністю, автор проявив достатнє знання предметної області і заслуговує оцінки «відмінно».


\vfill

\setstretch{1.0}
\begin{minipage}[t]{11cm}
\flushleft
Науковий керівник \\
професор кафедри обчислювальної математики,\\
доктор фізико-математичних наук
\end{minipage}
\begin{minipage}[t]{6cm}
\hfill
\flushright
Клюшин Д. А.
\end{minipage}

\end{document}
